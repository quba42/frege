\documentclass[12pt]{article}
\usepackage{frege}
\usepackage{amsmath}

%% For condensed itemize:
\newcommand{\squishlist}{
  \begin{list}{$\bullet$}{
    \setlength{\itemsep}{0pt}
    \setlength{\parsep}{0pt}
    \setlength{\topsep}{0pt}
    \setlength{\partopsep}{0pt}
    \setlength{\leftmargin}{1em}
    \setlength{\labelwidth}{1em}
    \setlength{\parskip}{0pt}
    \setlength{\partopsep}{0pt}
    \setlength{\rightmargin}{0pt}
    \setlength{\labelsep}{0.5em}}}
\newcommand{\squishlistend}{\end{list}}

\title{frege.sty \\ A \LaTeXe Package for Typesetting Begriffsschrift}
\author{Quirin Pamp \\ \texttt{Quirin.Pamp.2009@my.bristol.ac.uk}}

\begin{document}
\maketitle
\section{Background}
\subsection{begriff.sty}
  This package is based on begriff.sty released under the GNU General Public 
  License. 
  Copyright (C) 2003 by Josh Parsons (josh@coombs.anu.edu.au) with changes made in 
  October 2004 by Richard Heck (heck@fas.harvard.edu) and minor changes by Josh 
  Parsons to fix a problem with linespacing made in May 2005.

  While I could not have done so without the aforementioned work, I have reworked the 
  package from the ground up, to the point where some of the underlying approaches 
  have changed. On the downside this means there is no simple way of converting 
  anything typeset using begriff.sty to use this package instead. I felt this was 
  necessary to achieve a end result I was happy with.

\subsection{GNU General Public License}
  This package and all accompanying documentation is released under the GNU General 
  Public License. It is free software; you can redistribute it and/or modify it under 
  the terms of the GNU General Public License as published by the Free Software 
  Foundation; either version 2 of the License, or (at your option) any later version.

  This program is distributed in the hope that it will be useful, but WITHOUT ANY 
  WARRANTY; without even the implied warranty of MERCHANTABILITY or FITNESS FOR A 
  PARTICULAR PURPOSE. See the GNU General Public License for more details.

  You should have received a copy of the GNU General Public License along with this 
  program; if not, write to the Free Software Foundation, Inc., 59 Temple Place - 
  Suite 330, Boston, MA 02111, USA. (or just search for it online... Seriously.)

\subsection{Changes as Compared to begriff.sty}
  \squishlist
    \item correct relative lengths of the content stroke with respect to other 
      strokes attached to it;
    \item content strokes point at the middle of the followint symbols, rather 
      than the bottom;
    \item greater width for the assertion stroke as compared to the content stroke;
    \item a more intuitive structure for the conditional (arguments are now given in 
      the same order as they appear on the left of a completed formula);
    \item the command for the conditional with empty arguments now results in a 
      vertical line (conditional stroke) on it's own the other strokes are added as   
      the arguments;
    \item the linewidth is properly accounted for so that things remain properly 
      centered when scaled;
  \squishlistend

\subsection{Changes and Features yet to come}
  \squishlist
    \item incorporation of other frege related material such as fge.sty so as to put 
      all frege related material in a single package;
    \item a therefore stroke for typesetting arguments;
    \item a vertical shorthand stroke such as used by Frege for typesetting arguments 
      in ``Begriffsschrift'';
    \item reducing the length of the content stroke between the depressions of two 
      adjacent quantifiers;
  \squishlistend

\section{Features and Usage}
\subsection{Commands}
\subsubsection{Basic Commands}
  The following is a list of the basic commands provided by this package along with 
  accompanying output. The conditional is treated sepperately.
  \begin{flalign*}
    &\mbox{\textbackslash Fcontent} & &\Fcontent & & &\\
    &\mbox{\textbackslash Fncontent} & &\Fncontent\\
    &\mbox{\textbackslash Fnncontent} & &\Fnncontent\\
    &\mbox{\textbackslash Facontent} & &\Facontent\\
    &\mbox{\textbackslash Fancontent} & &\Fancontent\\
    &\mbox{\textbackslash Fanncontent} & &\Fanncontent\\
    &\mbox{\textbackslash Fquant\{a\}} & &\Fquant{a}\\
    &\mbox{\textbackslash Fnquant\{a\}} & &\Fnquant{a}\\
    &\mbox{\textbackslash Fnnquant\{a\}} & &\Fnnquant{a}\\
    &\mbox{\textbackslash Fquantn\{a\}} & &\Fquantn{a}\\
    &\mbox{\textbackslash Fquantnn\{a\}} & &\Fquantnn{a}\\
    &\mbox{\textbackslash Fnquantn\{a\}} & &\Fnquantn{a}\\
    &\mbox{\textbackslash Fnquantnn\{a\}} & &\Fnquantnn{a}\\
    &\mbox{\textbackslash Fnnquantn\{a\}} & &\Fnnquantn{a}\\
    &\mbox{\textbackslash Fnnquantnn\{a\}} & &\Fnnquantnn{a}\\
    &\mbox{\textbackslash Faquant\{a\}} & &\Faquant{a}\\
    &\mbox{\textbackslash Fanquant\{a\}} & &\Fanquant{a}\\
    &\mbox{\textbackslash Fannquant\{a\}} & &\Fannquant{a}\\
    &\mbox{\textbackslash Faquantn\{a\}} & &\Faquantn{a}\\
    &\mbox{\textbackslash Faquantnn\{a\}} & &\Faquantnn{a}\\
    &\mbox{\textbackslash Fanquantn\{a\}} & &\Fanquantn{a}\\
    &\mbox{\textbackslash Fanquantnn\{a\}} & &\Fanquantnn{a}\\
    &\mbox{\textbackslash Fannquantn\{a\}} & &\Fannquantn{a}\\
    &\mbox{\textbackslash Fannquantnn\{a\}} & &\Fannquantnn{a}\\
  \end{flalign*}
  This may seem like a daunting list, but there is an exceedingly simple way to think 
  about it. In a sense there are only two commands \textbackslash Fcontent and 
  \textbackslash Fquant\{\}. These two commands can be augmented with a combination 
  of as and ns so as to add assertion and negation strokes respectively. Any stroke 
  that is asserted (has a fat vertical line at the start) starts with `\textbackslash 
  Fa'. this may be followed by either one or two `n' to add one or two negation 
  strokes (the small vertical lines below the assertion stroke). Next comes the name 
  of the main command, either `quant' or `content'. Finally the quantifiers may be 
  followed by either one or two `n' to add one or two negation strokes to the 
  content stroke after the quantifier's depression.

  Consider also that many of these commands are only really present for completeness 
  sake. It is difficult to imagine a situation where a twice negated quantifier with 
  twice negated content would ever be needed.

  All quantifiers also have a mandatory argument that specifies the variable 
  associated with the quantifier. (Mandatory arguments are contained in a set of 
  curly brackets \{ and \}). This argument should be a single small letter and 
  will be typeset above the semi circular depression in the assertion stroke in 
  mathfrak font which is provided by tha amssymb package. This font can be used in 
  maths mode by using the command `\textbackslash mathfrak\{\}'. Note that all the 
  commands provided by this package may be used in both math and text mode. (Though 
  math mode usually results in better formatting.)

  Finally one may combine the above commands in arbitrary combinations which will 
  result in gapless longer strokes. (Eg.: $\Faquant{a}\Fnquantn{b}A$) which may be 
  roughly translated into english as ``for all $\mathfrak{a}$ there exists a 
  $\mathfrak{b}$ such that $A$''. (The commands I used for this expression are 
  \$\textbackslash Faquant\{a\}\textbackslash Fnquantn\{b\}A\$).

\subsubsection{Conditional}
  The conditional is the most important command in this package since it gives 
  Frege's Begriffsschrift it's two dimensional structure.\\
  The syntax for the Fconditional command is as follows:
  
  \textbackslash Fconditional[\textless option\textgreater]
  \{\textless consequent\textgreater\}\{\textless antecedent\textgreater\}\\
  The arguments may in principle be anything, but you will only get a begriffschrift 
  formula if the arguments are themselves given by appropriate commands from the list 
  of basic commands given earlier. As an example, an asserted conjunction between $A$ 
  and $B$ would be given as follows:

  \$\textbackslash Fconditional[\textbackslash Fancontent]\{\textbackslash Fncontent 
  A\}\{\textbackslash Fcontent B\}\$ \\
  and produce the following output: 

  $\Fconditional[\Fancontent]{\Fncontent A}{\Fcontent B}$

  In addition Fconditional may be nested as it's own argument to arbitrary depth. 
  Nesting in the option is not recommended. \\
  A conditional with nested consequent may be given as follows:

  \$\textbackslash Fconditional[\textbackslash Facontent]\{\textbackslash 
  Fcontent\textbackslash Fconditional\{\textbackslash Fcontent A\}\{\textbackslash 
  Fcontent B\}\}\{\textbackslash Fcontent\textbackslash Fcontent C\}\$\\
  and produces the following output:

  $\Fconditional[\Facontent]{\Fcontent\Fconditional{\Fcontent A}%
  {\Fcontent B}}{\Fcontent\Fcontent C}$\\
  A conditional with nested antecedent may be given as follows:

  \$\textbackslash Fconditional[\textbackslash Facontent]\{\textbackslash Fcontent 
  \textbackslash Fcontent A\}\{\textbackslash Fcontent\textbackslash Fconditional 
  \{\textbackslash Fcontent B\}\{\textbackslash Fcontent C\}\}\$\\
  and produces the following output:

  $\Fconditional[\Facontent]{\Fcontent\Fcontent A}{\Fcontent\Fconditional%
  {\Fcontent B}{\Fcontent C}}$

  Each section of a content stroke may thus be replaced with any of the strokes given 
  by the list of basic commands. Note that it is up to the user to place the 
  appropriate number of strokes in each argument to ensure that the content strokes 
  all line up on the right hand side.

\subsubsection{Brackets and Boxes}
  There are two more commands to be considered: 

  \textbackslash Fbracket\{\textless complex expression\textgreater\}

  \textbackslash Fbox\{\textless complex expression\textgreater\}\\
  Both Fbox and Fbracket take what I have called a `complex expression' for their 
  argument. A `complex expression' is any formula in begriffsschrift that has at 
  least one conditional in it. It is generally a good idea to put all complex 
  expressions into either a Fbox or a Fbracket. It is never necessary to place a 
  complex expressions into both an Fbox and an Fbracket since an Fbox simply is a 
  Fbracket without the actual brackets. Fbracket exists only for convenience with 
  the same effect being achieved by \textbackslash left(\textbackslash Fbox\{\} 
  \textbackslash right).

  The reason why the Fbox is a good idea, is that the baseline is very near the top of
  a complex expression of Begriffsschrift, which can make for some odd formatting 
  effects. In addition to placing the baseline at the middel of a complex expression 
  an Fbox ensures the expression is treated by LaTeX as a single object and given 
  enough space.

  Finally a complex expression may not format correctly in some environments (like 
  the align* environment for example) unless it is placed in an Fbox. In short, 
  always use an Fbox (or Fbracket).

\subsection{Lengths}
  In theory all the dimensions in this package can be changed with the command 
  \textbackslash setlength\{\textless name of length\textgreater\}\{\textless new 
  value\textgreater\}, though I have not done a great deal of testing and recommend 
  sticking to the default values. The following then is a table of all lengths: \\
  \begin{tabular}{|l|l|l|}
    \hline
    name & default value & description \\
    \hline
    \textbackslash Fbaselength & 10pt & half the length of a basic stroke \\
    \textbackslash Flinewidth & 0.5pt & the line width \\
    \textbackslash Fspace & 2pt & seperation between lines and text/formula \\
    \textbackslash Fassertwidth & 3\textbackslash Flinewidth & width of assert 
    stroke \\
    \textbackslash Fraiseheight & 1ex-\textbackslash Flinewidth & height of content 
    lines above baseline \\
    \textbackslash Fnegsep & 3\textbackslash Flinewidth & seperation between a 
    double negation \\
    \textbackslash Fnegshort & 2\textbackslash Flinewidth & space between negation 
    stroke and baseline \\
    \textbackslash Fquantwidth & 6pt & width of the semi-circular quantifier 
    depression \\
    \hline
  \end{tabular}
  the height of the conditional stroke is determined by the size of the contents of 
  the conditionals argument, as well as the baselineskip of the surrounding text. It 
  cannot be changed manually.

\subsection{Final Example}
  The Geach-Kaplan sentance (with thanks to Marcus Rossberg):
  \begin{align*}
    \Fbox%
    {%
      \Fconditional[\Fanquant{F}]%
      {%
        \Fcontent%
        \Fconditional%
        {%
          \Fcontent%
          \Fcontent%
          \Fcontent%
          \Fbox%
          {%
            f\Fbracket%
            {%
              \Fconditional[\Faquant{a}]%
                {\Fcontent C\mathfrak{(a)}}%
                {\Fcontent\mathfrak{F(a)}}%
            }%
          }%
        }%
        {\Fnquantn{b}\Fcontent\Fcontent\mathfrak{F(b)}}%
      }%
      {%
        \Fquant{c}%
        \Fquant{d}%
        \Fconditional%
        {%
          \Fncontent%
          \Fconditional{\Fcontent\mathfrak{c=d}}{\Fcontent\mathfrak{F(d)}}%
        }%
        {%
          \Fncontent%
          \Fconditional{\Fncontent\mathfrak{F(c)}}{\Fcontent A\mathfrak{(c,d)}}%
        }%
      }%
    }%
  \end{align*}
  And that is all. \\ 
  For comments, suggestions, identified errors, email me at \\
  \textless Q.Pamp.2009@my.bristol.ac.uk\textgreater.
\end{document}
